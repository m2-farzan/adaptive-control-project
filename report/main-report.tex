% !TEX TS-program = XeLaTeX
% Commands for running this:
%     xelatex main
% End of commands
\documentclass[a4paper, 12pt]{report}
\usepackage[top=30mm, bottom=35mm, left=25mm, right=25mm]{geometry}
\usepackage{fancyhdr, graphicx}
\usepackage{subfigure}
\usepackage[pagebackref=false,colorlinks,linkcolor=blue,citecolor=blue]{hyperref}
\usepackage{parskip}
\usepackage{amsmath}
\usepackage{amsthm}
\usepackage{mathtools}
\usepackage{amsfonts} 
\usepackage{pdflscape}

\usepackage{tikz}
\usepackage{pgfmath}
\usetikzlibrary{arrows.meta,arrows,angles,snakes}
\usepackage{multirow}
\usepackage{fancyvrb}
\usepackage{listings}
\lstset{
basicstyle=\small\ttfamily,
columns=flexible,
breaklines=true
}
\usepackage[section]{placeins}


%\usepackage[hidelinks]{hyperref}
\usepackage[extrafootnotefeatures, localise]{xepersian}
\settextfont[Scale=1]{XB Niloofar}
\setlatintextfont[Scale=0.9]{XB Niloofar}
%\setlatintextfont[Scale=0.9]{cm}
\linespread{1.6}
\newcommand{\مل}{\متن‌لاتین}
\newcommand{\tlr}[1]{\text{\lr{#1}}}
\newcommand{\ورب}[1]{\Verb!#1!}

\usepackage{threeparttable}
\newcommand{\جپانویس}{\tnote}

\usepackage{comment}
\usepackage{titling}
\pagestyle{fancy}
\fancyhead[L]{}
\fancyhead[R]{\it{\thetitle}}
\setlength{\headheight}{27pt}

\newcommand{\آ}{\emph{%
پارک}}
\newcommand{\ب}{\emph{%
چین}}
\newcommand{\پ}{\emph{%
شجاعی}}

\عنوان{گزارش پروژه پایانی درس کنترل تطبیقی}
\نویسنده{محمد فرزان (99743297) \\ عرفان محمودتبار (99743279)}
\تاریخ{\فضای‌و{\پر} نسخه ۱ | \امروز}
\شروع{document}
{\centering
به نام خدا
\vspace{2cm}

\hspace{0.7cm}
\includegraphics[height=2.5cm]{img/iust-nomech.pdf}

دانشکده مهندسی مکانیک

\vspace{0cm}

{\let\newpage\relax\maketitle}}

\صفحه‌جدید

\chapter*{فهرست تغییرات}
\شروع{لوح}[h]
\تنظیم‌ازوسط
\شروع{جدول}{p{2cm}p{2.5cm}p{8cm}}
شمارهٔ نسخه & تاریخ تحویل & تغییرات \\
\hline
۱ & ۱۴۰۰/۴/۲۸ & [اولین نسخه]
\پایان{جدول}
\پایان{لوح}


\صفحه‌جدید
\فهرست‌مطالب

\فصل{مقدمه}
\ورودی{intro}
\فصل{تشریح مقالات مورد بررسی}
\ورودی{overview}
\فصل{پیاده‌سازی نرم‌افزاری}
\ورودی{implementation}
\فصل{تحلیل نتایج}
\ورودی{results}
\فصل{بحث و نتیجه‌گیری}
\ورودی{summary}
\پیوست

\فصل{نقش اعضا}
عرفان محمودتبار:
\شروع{فقرات}
\فقره پیاده سازی کد شبیه ساز مدل ربات
\فقره نگارش گزارش مقاله \آ\ (بخش مدل ربات)
\فقره نگارش گزارش مقاله \ب
\فقره نگارش سایر قسمت های گزارش 
\پایان{فقرات}

محمد فرزان:
\شروع{فقرات}
\فقره پیاده‌سازی کد شبیه‌ساز کنترلر
\فقره پیاده‌سازی کد تهیه نمودارها
\فقره نگارش گزارش مقاله \آ\ (بخش کنترلر)
\فقره نگارش گزارش مقاله \پ
\پایان{فقرات}

\chapter*{مراجع}
\pagestyle{empty}
{
%\bibliographystyl{acm-fa}%{chicago-fa}%{plainnat-fa}%
\begingroup
\renewcommand{\chapter}[2]{}%
\lr{
%\bibliographystyle{ieeetr-fa}
\bibliographystyle{vancouver}
\bibliography{ref}
}
\endgroup
}
\پایان{document}
