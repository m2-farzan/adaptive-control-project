\قسمت{بحث دربارهٔ نتایج}
در این پروژه مقاله اول (مقاله \آ) پیاده سازی شد که همانطور که در بخش تحلیل نتایج مشاهده می‌‌‌‌‌‌شود، تکرار نتایج مقاله با موفقیت انجام شد.\\
همچنین روش کنترلی ارائه شده برای مسیر های پیچیده تر از آنچه در مقاله استفاده شد نیز تست گردید که نتایج بدست آمده نشان می‌دهد، این روش کنترلی برای مسیر های پیچیده تر خوب عمل نخواهد کرد.\\
در مقاله پیاده سازی شده، هدف اصلی، طراحی کنترل کننده مسیر به گونه ای بود که ربات مسیر از پیش تعیین شده ای را دنبال کند و مسائلی همچون نحوه داده برداری مسیر در آن مورد بررسی قرار نگرفت. این قابلیت با پیاده سازی مقاله \ب\ اضافه می‌شود. همچنین در صورت پیاده سازی مقاله \پ\ عملکرد های غیر خطی کنترلر مانند اشباع عملگر ها که در مقاله \آ\ مورد بررسی قرار نگرفت در نظر گرفته می‌شود.  
\قسمت{پیشنهادها}
در پایان  پیشنهاد های مطرح شده در این تحقیق به شرح زیر می‌باشد:
\شروع{فقرات}
\فقره اضافه کردن جدول فهرست علائم اختصاری در مقاله \آ
\فقره انتشار کد استفاده شده در شبیه سازی در مقاله \آ
\پایان{فقرات}
