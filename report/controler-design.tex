% !TeX root = main-report.tex
طراحی کنترلر تطبیقی با استفاده از روش کنترل سطح دینامیکی\پانویس{Dynamic Surface Control (DSC)} انجام شده است.
برای این کار ابتدا سه سطح کنترلی $S_1$ تا $S_3$ تعریف شده است. تعریف اولین سطح خطا به صورت زیر است:
\begin{equation}
	S_1 = R_z(\theta) (q - q_r)
\end{equation}
که در معادلهٔ فوق، $R_z(\theta)$ ماتریس دوران حول محور قائم جهانی می‌باشد.

برای کنترل ربات، باید مقدار خطای سطح $S_1$ به صفر برسد.
بر این اساس، مقدار مطلوب سرعت خطی و سرعت زاویه‌ای ربات محاسبه شده و با نماد $x_{2f}$ نمایش داده شده است.
سپس، خطای این متغیر از مقدار مطلوب به صورت $S_2$ تعریف شده است:
\begin{equation}
	S_2 = z - x_{2f}
\end{equation}
به طور مشابه، کنترل این سرعت‌ها باید به واسطهٔ کنترل جریان موتورهای الکتریکی انجام شود.
لذا اگر این جریان‌ها به عنوان ورودی در نظر گرفته شوند، مقدار مطلوب $x_{3f}$ برای آن‌ها به دست می‌آید که اختلاف آن با مقدار فعلی خود یک سطح خطای دیگر است:
\begin{equation}
	S_3 = i_a - x_{3f}
\end{equation}
در نهایت، کنترل این سطح خطا با استفاده از ولتاژ اعمال شده به موتورهای الکتریکی انجام می‌شود.
در قانون کنترلی در نظر گرفته شده برای هر یک از این سطوح، یک متغیر تطبیقی به روز شونده وجود دارد. بدین ترتیب، در کل چهار متغیر اسکالر تطبیقی
$\hat{a}_1$ تا $\hat{a}_4$
خواهیم داشت.
رابطهٔ این متغیرها با تخمین مقادیر فیزیکی در معادلات زیر آمده است:
\begin{align}
	\hat{a}_1 \simeq 1/r
\\
	\hat{a}_2 \simeq R/r
\\
	\hat{a}_3 \simeq || W_1 ||^2
\\
	\hat{a}_4 \simeq || W_2 ||^2
\end{align}

که در معادلات فوق:
$$
	W_1 = \left[
	\begin{matrix}
		\dfrac{r^3 m_c d}{4R^2n_1k_{t_1}} &
		\dfrac{r^3 m_c d}{4R^2n_2k_{t_2}} &
		\dfrac{d_{11}}{n_1 k_{t_1}} &
		\dfrac{d_{22}}{n_2 k_{t_2}} &
		\dfrac{m_{11}}{n_1 k_{t_1}} &
		\dfrac{m_{12}}{n_1 k_{t_1}} &
		\dfrac{m_{11}}{n_2 k_{t_2}} &
		\dfrac{m_{12}}{n_2 k_{t_2}} &
		\dfrac{d_{m1}}{n_1 k_{t_1}} &
		\dfrac{d_{m2}}{n_2 k_{t_2}}
	\end{matrix}
	\right]^T
$$
$$
W_2 = \left[
\begin{matrix}
	r_{a_1} & r_{a_2} & n_1 k_{e_1} & n_2 k_{e_2} & l_{a_1} & l_{a_2}
\end{matrix}
\right]^T
$$

\موکد{توجه: معنای نمادهای ریاضی موجود در معادلات فوق، در بخش مدل‌سازی همین گزارش شرح داده شده است.}

همانطور که دیده می‌شود، عبارت‌های سمت راست پارامترهای مورد تخمین در برخی موارد از ترکیب چندین پارامتر فیزیکی تشکیل شده‌اند.
در حقیقت، سیستم روی هم رفته ۱۶ پارامتر مجهول دارد اما در کنترلر طراحی شده فقط ۴ پارامتر تخمین زده می‌شود که ترکیب‌هایی از ۱۶ پارامتر اصلی هستند.
عملا چون پارامترها در این ترکیب‌ها در مدل ظاهر می‌شوند، احتیاجی به تخمین زدن مجزای آن‌ها نیست و با این راهکار می‌توان مجهولات کنترلر را به فضایی با ابعاد کمتر محدود کرد.
بدین ترتیب، از پیچیدگی معادلات نیز کاسته می‌شود و همچنین خطر برازش ناخواسته نویز به عنوان پارامتر نیز کاهش می‌یابد.

همچنین لازم به ذکر است که در قانون کنترلی به کار رفته برای کنترل دو سطح خطای $S_2$ و $S_3$، ورودی کنترلی از فیلتر پایین‌گذر نیز عبور می‌کند (به ترتیب با ثابت زمانی $\tau_1$ و $\tau_2$).
دلیل این تصمیم طراحی در مقاله ذکر نشده است اما به نظر می‌رسد این انتخاب با هدف کاهش اثر نویز بوده است.
علت این که از فیلترهای پیچیده‌تر استفاده نشده است، احتمالاً این است که فیلتر پایین‌گذر در فضای زمان معادلات را پیچیده نمی‌کند و برای اثبات پایداری لیاپانوف مشکلی به وجود نخواهد آورد.

در‌ مقاله همچنین پایداری سیستم حلقه بسته با استفاده از روش لیاپانوف ثابت شده است.
