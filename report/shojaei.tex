% !TeX root = main-report.tex
\زیرقسمت{پیش‌زمینه}
این مقاله بر پایهٔ روش \مل{DSC} توصیف شده در \آ\ می‌باشد. با این حال، در روش \آ\ رفتار غیرخطی عملگرها نظیر اشباع مدل نشده است و این امر باعث می‌شود رفتار حالت گذرای ربات غیر بهینه باشد. همچنین، نگه داشتن عملگرها در حالت اشباع در بلند مدت می‌تواند منجر به کاهش عمر آن‌ها شود. به همین دلیل، در این مقاله تلاش شده است با استفاده از روش‌هایی که در ادامه آمده است، عملکرد غیرخطی کنترلرها در نظر گرفته شود.

\زیرقسمت{نوآوری‌ها}
برای رفع مشکلات فوق، این پروژه یک مشاهده‌گر غیر خطی را به کار برده است. همچنین، با استفاده از تابع تانژانت هیپربولیک، رفتار اشباع مدل شده است و در نهایت، یک شبکهٔ عصبی در طراحی کنترلر استفاده شده است.

در این مقاله نیز ابتدا یک مدل مرتبه دوم برای ربات پیشنهاد شده است. این ربات مدل برای طراحی کنترلر و همچنین تحلیل پایداری لیاپانوف استفاده شده است. همچنین، در پایان مقاله یک شبیه‌سازی عددی انجام شده و نتایج آن برای صحت‌سنجی با موفقیت استفاده شده است.

شبکهٔ عصبی مورد استفاده برای برازش توابع غیرخطی واقعی مورد استفاده قرار گرفته است و از نوع \مل{RBFNN}\پانویس{Radial basis function neural network} می‌باشد. مانند اغلب موارد کاربرد \مل{RBFNN}، شبکه یک لایهٔ پنهان داشته و تابع تبدیل در آن گاوسی است. دلیل انتخاب این شبکهٔ عصبی به طور صریح ذکر نشده است اما احتمالاً به دلیل توانایی بالای این شبکه به تخمین توابع پیوستهٔ نرم با تعداد نقاط ورودی و اندازهٔ شبکهٔ کوچک می‌باشد؛ چرا که مدل این شبکهٔ عصبی به گونه‌ای است که ورودی‌هایی که فاصلهٔ کمی داشته باشند، خروجی مشابهی دریافت می‌کنند. ورودی این شبکهٔ عصبی به صورت زیر است:
\begin{equation}
	x_w = \left[
	\begin{matrix}
		q \\ \dot{y}_r \\ \ddot{y}_r
	\end{matrix}
	\right]
\end{equation}
که این ورودی‌ها خود از مشاهده‌گر غیرخطی دریافت می‌شوند و در آن‌ها
$q = [\zeta, \phi, \gamma_1, \gamma_2]^T$
است که مولفه‌های آن به ترتیب بردار مکان مرجع، هدینگ ربات و زوایای فرمان‌گیری را نشان می‌دهد و همچنین
$y_r = [\zeta + R_z(\phi) P_R, \gamma_1 ]$
می‌باشد که در آن $R_z$ ماتریس دوران حول محور جهانی قائم و $P_R$ بردار مکان واقعی ربات می‌باشد.

خروجی شبکه نیز عبارت غیر خطی غیر قطعی می‌باشد که به تانژانت هیپربولیک اضافه می‌شود و در هر گام زمانی با معادلهٔ زیر به دست می‌آید:
\begin{equation}
	-M_2(q) \ddot{y}_r - C_2 (q, \dot{y}_r) \dot{y}_r - D_2(q) \dot{y}_r
\end{equation}
جایی که $M_2$، $C_2$ و $D_2$ بر اساس مدل فیزیکی تعریف شده‌اند.

مزایای نهایی روش پیشنهاد شده در این مقاله به صورت زیر فهرست می‌شود:
\شروع{فقرات}
\فقره کاهش ریسک اشباع عملگرها که به عملکرد کنترلی بهتر در حالت گذرا و طول عمر بیشتر اجزا منجر می‌شود.
\فقره نیاز نداشتن به اندازه‌گیری سرعت به دلیل استفاده از مشاهده‌گر غیرخطی
\فقره عمومی بودن مدل برای ربات‌های چرخ‌دار با درجات پویایی\پانویس{mobility} و فرمان‌پذیری\پانویس{steerability} مختلف
\فقره پیچیدگی کمتر در مقایسه با روش \مل{Back-stepping}
\فقره مقاومت در برابر عدم قطعیت پارامترها، اغتشاشات خارجی و اصطکاک
\پایان{فقرات}