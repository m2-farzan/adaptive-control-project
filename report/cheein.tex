\زیرقسمت{پیش زمینه} 
در پیاده سازی کنترل کننده های مسیر در ربات های متحرک در کاربرد های واقعی، لازم است مسیر از پیش تعیین شده جهت تعقیب، نمونه برداری شود. چگونگی انجام این کار از اهمیت ویژه ای برخوردار است چرا که به عنوان مثال، فاصله میان هر دو نقطه متوالی از مسیر جهت تعقیب، بر پایداری، عملکرد کنترلر و پدیده هایی مانند اشباع عملگرهای ربات تاثیر مستقیم خواهد داشت.\\
در این مقاله سعی می‌شود با ارائه روش هوشمند جهت نمونه برداری، عملکرد کنترلر بهبود بخشیده شود. روش ارائه شده محدود به کنترلر خاصی نمی‌شود و قابل استفاده در تمامی کنترلر های ارائه شده در منابع می‌باشد.\\

\زیرقسمت{نوآوری}
جهت دست یابی به اهداف بیان شده در بخش قبل، ابتدا مسیر 
$
\Omega
$
به گونه ای گسسته سازی می‌شود که فواصل هر دو نقطه متوالی با هم برابر باشد(
$
||P_i - P_{i+1}||_2 = constant
$
).
سپس از هر نقطه، از 
$
\delta = i+1
$
تا آخرین نقطه مسیر 
$
\Omega
$،
مقدار 
$
||P_i - P_{\delta}||_2
$
محاسبه می شود. جهت انتخاب بهینه توابع هزینه به فرم زیر تعریف و برای هر نقطه محاسبه می‌شود:
\begin{equation}
	f = {v_{k+1}}^2 + {\psi_{k+1}}^2  
\end{equation}
\begin{equation}
	g = {\Delta_{x}}^2 + {\Delta_{y}}^2 = ||P_i - P_{\delta}||_2
\end{equation}
که در آن 
$
({v_{k+1}}^2 , {\psi_{k+1}}^2)
$
فرمان های کنترلی متناظر می‌باشد. 
سپس نقطه ی بهینه ای که از نظر سینماتبکی با موقعیت فعلی ربات تطابق داشته باشد به عنوان نقطه بهینه انتخاب می‌شود. جهت بررسی تطابق سینماتیکی بررسی می‌شود که شعای انحنای میان موقعیت فعلی و نقطه انتخاب شده از حداکثر مقدار تعیین شده، بیشتر نباشد.\\
برای جلوگیری از برخورد ربات با اجسام متحرک موجود در محیط، از روش تخمین احتمال برخورد استفاده می‌شود که در آن به هر برخورد احتمالی، تابع احتمال گاوسی زیر اختصاص داده می‌شود:
\begin{equation}
	{p}_{(P_r,P_0)}=\dfrac{exp(-\dfrac{1}{2}(A-B))\sqrt{|\gamma|}}{\sqrt{|\gamma_r|}\sqrt{|\gamma_0|}}
\end{equation}  
که در آن 
$
\gamma_r
$
ماتریس کواریانس موقعیت ربات
$
P_r
$
،
$
\gamma_0
$
ماتریس کواریانس مربوط به موقعیت جسم تشخیص داده شده
$
P_0
$
و 
$
A-B
$
توابعی بر حسب 
$P_0$
،
$P_r$
،
$\gamma_0$
و
$
\gamma_r
$
می‌باشد. سرانجام احتمال برخورد در هر لحظه با 

$
N
$
مانع شناسایی شده 
از رابطه زیر محاسبه می‌شود:
\begin{equation}
	{p}_{P_{r,t}}=1/\eta \sum_{i=1}^{N} p(P_r,P_{P_{o,i}})
\end{equation} 
اگر احتمال برخورد از مقدار 
$
0.9
$
بیشتر شود، ربات توقف خواهد کرد و این توقف تا زمانیکه این احتمال به زیر 
$
0.9
$
کاهش نیابد ادامه خواهد داشت.\\
موضوعی که در این مقاله به آن پرداخته می‌شود را می توان به طور خلاصه این طور بیان کرد:\\
با دانستن مسیر از پیش تعیین شده 
$
\Omega
$
و موقعیت لحظه 
$
P_k : k
$
، نقطه 
$
[x_{ref,k+1}, y_{ref,k+1}]^T \in \Omega
$
به گونه ای انتخاب شود که: ۱) عملکرد کنترلر بهینه باشد (کترین خطای ردیابی و تلاش کنترلی) ۲) از اشباع شدن عملگر ها جلوگیری شود ۳) تطابق سینماتیکی با ربات داشته باشد ۴) از برخورد با اجسام متحرک موجود در محیط جلوگیری کند.
