\قسمت{تعریف پروژه}
در این پروژه درسی، سه مقاله از دو مجله \مل{IEEE TRANSACTIONS} و \مل{INTERNATIONAL JOURNAL OF ADAPTIVE CONTROL AND SIGNAL PROCESSING}
با موضوع طراحی کنترل کننده تطبیقی برای ربات های زمینی چرخ دار به گونه ای انتخاب شدند که مقالات در امتداد هم بوده و هر کدام نوآوری مخصوص به خود را داشته باشد.\\
سه مقاله مورد بحث در سراسر این متن به ترتیب با نام های \آ\پانویس{Park}، \ب\پانویس{Cheein} و \پ\پانویس{Shojaei} مورد اشاره قرار گرفته اند و شماره ارجاع به آن ها به ترتیب \مرجع{park2009simple}، \مرجع{cheein2015intelligent} و \مرجع{shojaei2015neural} می‌باشد (مقاله ۲ و ۳ به مقاله ۱ ارجاع داده اند). 
\قسمت{مقدمه‌ای بر مقالات مورد بررسی}
مقاله اول مدل تطبیقی پایه را بر اساس روش \مل{DSC}\پانویس{Dynamic surface control} ارائه داده است که پایداری آن را نیز با استفاده از روش لیاپانوف \پانویس{Lyapunov} بررسی کرده است. این مقاله به عنوان مقاله اصلی مورد بررسی در این تحقیق می باشد. در مقاله دوم سعی شده است با ارائه روش هوشمند برای نمونه برداری، عملکرد مدل بهبود داده شود. در مقالۀ سوم، تخمین زن تطبیقی از حالت خطی به شبکه عصبی تغییر داده شده است تا پدیدە های غیر خطی مانند اشباع عملگرها بهتر مدل شوند.\\
در روش های مورد بررسی علاوه بر سینماتیک و دینامیک بیرونی ربات، رفتارهای دینامیکی عملگرهای مکاترونیکی داخلی آن نیز مدل شده است که باعث می‌شود کنترلر ها عمکرد بهتری داشته باشند.  

\قسمت{بخش‌های این گزارش}
این گزارش در ۵ فصل تهیه شده است:
\شروع{فقرات}
\فقره[] \متن‌سیاه{فصل ۱ - مقدمه}: فصل حاضر.
\فقره[] \متن‌سیاه{فصل ۲ - تشریح مقالات مورد بررسی}: در این فصل مقاله مورد نظر برای بررسی به طور دقیق و دو مقاله دیگر به طور اجمالی مورد بررسی و مقایسه قرار گرفتند.
\فقره[] \متن‌سیاه{فصل ۳ - پیاده‌سازی نرم‌افزاری}: در این فصل به الگوها و کتابخانه‌های مورد استفاده برای بازتولید الگوریتم‌های آمده در مقالات اشاره شده است. همچنین در این فصل فهرست فایل‌ها و همچنین راهنمای اجرای کد آمده است.
\فقره[] \متن‌سیاه{فصل ۴ - تحلیل نتایج}: در این فصل نتایج اجرای کد‌ها در قالب نمودارها و جداول آمده است.
\فقره[] \متن‌سیاه{فصل ۵ - بحث و نتیجه‌گیری}: در این فصل با توجه به نتایج به دست آمده، مقالات نقد شده‌اند و پیشنهاداتی نیز آمده است.
\پایان{فقرات}

