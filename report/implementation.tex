\قسمت{زبان برنامه‌نویسی}
برای بازسازی الگوریتم های استفاده شده در این پروژه از زبان برنامه نویسی متلب\پانویس{Matlab} استفاده شده است.
\قسمت{راهنمای اجرای فایل‌ها}
در پیوست یک فایل فشرده قرار دارد که کد مربوط به این پروژه در آن موجود است. جدید ترین نسخه این کد در مخزن آنلاین گیتاب ذیل آدرس 
\url{https://github.com/m2-farzan/adaptive-control-project}
نگهداری می‌شود.\\
فایل اصلی \ورب{main.m} نام دارد که با اجزای آن تمام نمودار های اصلی استفاده شده در گزارش تولید و به صورت فایل های \ورب{eps} در کنار فایل \ورب{main.m} ذخیره می‌شود. البته، فهرست تمام فایل‌ها و نقش آن‌ها نیز در جدول \رجوع{کلید} آمده است.
\شروع{لوح}[h]
\تنظیم‌ازوسط
\شرح{کلید فایل‌ها به ترتیب اهمیت}

\شروع{جدول}{چر}
\multicolumn{1}{c}{نام فایل} &
\multicolumn{1}{c}{نقش} \\
\hline
\hline
\verb|main.m| & اجرای شبیه سازی و تولید نمودارها \\
\verb|Xdot.m| & تحول متغیر های حالت بخش مدل ربات\\
\verb|Wdot.m| & تحول متغیر های حالت بخش کنترلر\\
\verb|Sdot.m| & تحول متغیر های حالت مدل ربات و کنترلر\\
\verb|load_plant_params.m| & مقدار دهی اولیه پارامتر های مربوط به بخش مدل ربات \\
\verb|load_controller_params.m| & مقدار دهی اولیه پارامتر های مربوط به بخش کنترلر \\
\verb|load_simulation_params.m| & مقدار دهی اولیه پارامتر های مربوط به شبیه سازی \\
\verb|generate_disturbance.m| & تولید اغتشاش استفاده شده در شبیه سازی\\
\verb|load_initial_state.m| & مقدار دهی اولیه متغیر های حالت سیستم و کنترلر \\
\verb|z_r.m| & تولید سیگنال های مرجع \\	
\verb|visualize_results.m| & رسم نمودار ها \\
\verb|report/*| & سورس این گزارش \\
\verb|scratchpad/*| & تست‌های مربوط به فرایند توسعه
\پایان{جدول}
\برچسب{کلید}
\پایان{لوح}


